\documentclass[a4paper]{cctart}
\usepackage{amsmath,bm,cases,cchead,fancyhdr}
\usepackage[affil-it]{authblk}
\usepackage{scrextend}
%\ffote[1.2em]{1.2em}{0.5em}{\thefootnotemark.\space}




\renewcommand{\vec}[1]{\bm{#1}}
\topmargin=0pt \oddsidemargin=0.66cm \evensidemargin=0.66cm
\textwidth=14cm \textheight=21cm
\renewcommand{\theequation}{\thesection\.arabic{equation}}
\newcommand{\eq}[1]{\begin{equation}{#1}\end{equation}}
\numberwithin{equation}{section} \pagestyle{fancy}
\markboth{清华大学文化素质讲座课程报告}{课程论文}
\title{文化素质讲座课程论文}
\author{高宸}
\date{2015.12}
\affil{电子工程系~无22班~2012011027}


\begin{document}
\maketitle
\begin{abstract}
文化素质讲座是一门创造性的课程,经过七个学期的学习,八个内容丰富、精彩绝伦的讲座,让我收获颇丰
。那么,接下来,
就让我来回顾下这八次美妙的旅程。
\end{abstract}

\section{内容综述}\label{1}
\subsection{贫富差别与社会公平}
清华大学社会学院院长李强给我们带来了名为“贫富差别与社会公平”的讲座,深刻阐述了对于贫富差距、
社会公平公正的理解。在讲座中,李强老师指出,对于公平公正的理解,有两种观念,一种观点认为,社会
分层结构本身就是不公正的,要实现公平必须消除分层结构,而另一种观点认为,分层结构的差异是无法消
除的,公平公正旨在保证人们进入这些机构的渠道和机会是公平公正的。

对于贫富差距的问题,李强老师介绍了衡量贫富差距的一个重要标准--基尼系数,基尼系数越大,贫富差距
就越大,目前中国的基尼系数大于0.5,表明目前还存在着较为严重的贫富差距问题。我们要正视贫富差距问
题,要警惕极少人垄断财富的极端,也要警惕民粹主义。

李强老师认为,中华民族的未来是广大中国民族的未来,不能是被少数人垄断财富的未来。
\subsection{中国传统文化价值理念的现代意义}
中国文化研究所所长刘梦溪给我们带来了名为“中国传统文化价值理念的现代意义”的演讲。在讲座中,刘
老师首先为我们讲述了中国传统文化在不同历史段落中的不同特色,先秦有诸子百家,在百家争鸣的时代,
产生了大量活跃的思想家,他们的思想涉及到世界观、价值观;秦汉时期,中国的学术思想主要表现为制度
文化,其中汉代以“经学”作为主要的学术思想,独尊儒术;到了三国魏晋南北朝时期,『玄学』成为主要学
术思想;隋唐时期,佛学发展到鼎盛时期;宋代的理学思想可谓集大成者,吸收了前代的各种思想,将中国
思想推向了一个高峰;王阳明的『心学』是明代的代表思想之一;清代的学术思想主要是考据学,体现朴实
无华的治学风格。

刘老师指出,中华传统文化中的价值理念永远不会过时,我们要集成并弘扬优秀的价值理念,视之为中华民
族的宝贵遗产。

\subsection{金融危机与乡村建设}
中国人民大学温铁军教授给我们带来了名为“金融危机与乡村建设”的演讲。早就耳闻温老师的名字,他在
经济建设颇有建树,这次讲座也是围绕农村经济建设展开。

温教授首先讲到,为什么中国在全球金融危机的背景中一直屹立不倒,“风景这边独好”,和中国雄厚的实
质资产是分不开的,中国的实质资产仍然处于扩张状态,金融资产尚未泡沫化。而现代化的背后,是代价,
虽然表面上我们只看到了利益,但是潜在着不可见的不安全因素。

对于乡村建设,温教授认为,农村问题是中国经济发展的重头戏,社会主义新农村建设是当前国家的一项重
大战略,但是目前处于攻坚阶段,遇到了很多难以解决的困难与挑战,但是,决不能因此就忽视“三农”问
题,对于农村建设绝不能动摇。
\subsection{中国说唱艺术的魅力}
著名相声表演艺术家姜昆老师给我们带来了名为“中国说唱艺术的魅力”的讲座。作为中国曲艺界的领军人
物,姜昆老师对于曲艺的定位是劳动的产物,即曲艺产生于最广大的劳动人民,所谓的曲艺,即是中华民族
各种“说唱艺术”的总称,中国的说唱艺术,异彩纷呈,包括丰富多样的艺术形式,“说”包括相声、快板、
评话等,“唱”包括评弹、琴书、大鼓等等。

姜昆老师给我们讲了他在曲艺创作中的一些趣事,并强调曲艺创作要深入群众,因为这是群众的艺术。在讲
座现场,姜昆老师给我们表演了《八扇屏》的一段,展现了扎实的艺术功底,《八扇屏》是一部运用“贯口”
手法(对口相声中常见的表现形式,也叫“背口”)的传统相声经典作品。

姜昆老师同时也讲到与时俱进的重要性,时代是发展的,曲艺艺术也要向前看,广大的曲艺艺术工作者,应
该努力创作,继承发扬中华民族的优秀艺术。
\subsection{电影创作中的历史观与人性表达}
著名导演冯小宁带来了名为“电影创作中的历史观与人性表达”的讲座。冯小宁先生是中国第五代导演的杰
出代表,从事电影创作几十年,自然有极深的体会。

冯老师认为,电影中体现的价值观会很直接地传递给观众,作为创作者,对于电影的价值观传达,自然必须有
自己的考量。冯老师的作品,很多反映近代的民族历史,有著名的『战争三部曲』,这些作品,并不仅仅停留
在叙事上,更从独特的视角,为我们展现一个个可歌可泣的民族传说。其背后,体现了对于战争、和平的思考。
冯小宁老师的电影,往往从民族冲突角度出发表现人性冲突,揭示了人类灵魂的特质。冯老师认为,电影是死
的,但是创作者是活的。

冯老师认为,只有真正用心去创作电影,观众才会真正地被电影所感染。
\subsection{深度学习}
清华大学工程力学系郑泉水教授给我们带来了“精深学习”的精彩演讲。讲座的核心是,如何提高清华学子
的综合素质与研究创新能力,郑教授以丹尼尔·科伊尔的一万小时天才理论\footnote{该理论认为,要成为
某个领域的专家,需要10000小时的练习}引入主题,认为在信息爆炸的今天,精深学习的概念显得尤为重要
。所谓精深学习,即是不再简单地将大量时间投入到某项工作中,而是在一定条件下,不断挑战自我,不断
犯错以获得进步和提高。

郑教授以自己的学习经历、爱因斯坦和比尔盖茨的成才经历做横向比较,具体说明了如何进行精深学习,以
及精深学习有怎样的成效。郑教授认为,清华的学生不能将时间浪费在刷学分绩上,而应该心怀的远大的理
想,在热爱的领域,展开挑战性的学习。而要做到这一点,也需要学校建立更完善的课程体系与评价体系。
郑教授认为知识有三重境界,由浅入深分别是:信息、技能、态度。知识只有是“活”的才有意义,“死”
的知识反而会束缚思维。郑老师以钱学森力学班的实践为例进一步说明了这一点。

在讲座结束后的互动环节,几位同学提出了很有前瞻性的问题,比如一位同学问是否应建立并推广免修制度,
而据我所知,当笔者写下本文时,已有部分院系对于部分课程实施了免修制度,这位同学的问题如今已经成
为了现实。
\subsection{美好青春我做主——青春红丝带校园行}
首都高校青春红丝带社团工作领导小组办公室副主任许建农先生给我们带来了关于艾滋病与毒品的讲座,此
次讲座针对于目前愈演愈烈的大学生艾滋病问题,普及艾滋病预防知识,让我们正确认识艾滋病。同时,在
很多时候,艾滋病是作为毒品的“衍生品”传播,所以徐老师细致讲解了毒品的种类、危害及其与艾滋病的
关系。

许老师指出,近年来,我国的艾滋病感染者和病人数量连年攀升,另一件恐怖的事实是,高校学生的艾滋病
感染者和病人数量的增加速率远高于平均速率,而这其中大部分是通过性传播,尤其是男男性传播。艾滋病
之所以特殊,因为艾滋病带来的不仅仅是病痛,更有白眼与冷落。对待艾滋病感染者,我们要给予足够的尊
重,保护隐私显得尤为重要,需要卫生行政部门、学校行政部门的共同努力。作为一名大学生,首先要充分
认识艾滋病的危害,洁身自好,遏制源头,同时也要提防一切误感染的可能性。

同时,徐老师为我们讲解了常见毒品的相关知识,毒品似乎离我们很远,其实很近,对待毒品,必须采取零
容忍的态度,因为传播、使用毒品,害人害己,天理不容。
\subsection{一带一路——历史视野与现实关照}
清华大学历史系教授张国刚教授为我们带来了关于“一带一路”的精彩讲座,此次讲座举办适逢我选修《党
的知识概论》,所以对于此次讲座有着特殊的感受。

所谓“一带一路”,即“丝绸之路经济带”与“21世纪海上丝绸之路”,由中国国家主席习近平在2013年出
访中亚和东南亚国家期间提出,受到国际社会的广泛高度关注。起始于古代中国的古代路上商业贸易路线“
丝绸之路”连接亚洲、非洲和欧洲,是东方与西方之间经济、政治、文化进行交流的重要道路;而“海上丝绸
之路”是古代中国与外国交通贸易和文化交往的海上通道,该路主要以南海为中心,所以又称“南海丝绸之路
”。

张老师的讲座内容对应于标题,从“历史视野”与“现实关照”两部分分别剖析“一带一路”。从历史上讲,
“一带一路”的提出不是信口开河,也不是纸上谈兵,它是一个可靠而且具有重大意义的战略,承接了古代中
国,具有深厚的人文关怀。同时,今非昔比,“一带一路”战略也与当今的时代背景相契合,从国内的角度,
目前改革步入深水区,到了矛盾集聚、风险挤压的关键期;从国外的角度,世界经济全球化、区域经济一体化
加快推进,世界经济同样到了转型升级的关键阶段。

\DeclareRobustCommand\CTeX{$\mathbb{C}$\kern-.05em\TeX{}}

\section{心得体会}\label{2}
\subsection{本构方程的必要性}


\subsection{$Noll$三原则}

(1){\kaishu 物质客观性原理}
(2){\kaishu 确定性(遗传性)原理}
(3){\kaishu 局部作用原理}


\section{课程意见和建议}\label{3}



\subsection{简单物质的本构方程}


\section{结论}

\section{致谢}
在此本人表示感谢!

\begin{thebibliography}{9}
\bibitem{Gurtin}Gurtin~M~E, 1981, An Introduction to Continuum
Mechanics, Academic Press.
\bibitem{kzb1}Kuang~Z~B, 1990, Integral constitutive equation
of elastic-plastic materials. Acta Mechanica Solida
Sinica,3:245-262.
\bibitem{Miller}Miller~A~K, 1987, Unified constitutive equations
for creep and plasticity. Elsevier Applied Science World
Publishing Coporation, New York.
\bibitem{Tanaka}Tanaka~T~G, Miller~A~K, 1988, Development of a
method for integrating time-dependent constitutive equations with
large, small or negative strain rate sensitivity. Int J Numerical
Methods in Engng, 26:2457-85.
\bibitem{kzb2} 匡震邦,1989,非线性连续介质力学基础,西安交通大学出版社.
\end{thebibliography}
\end{document}
