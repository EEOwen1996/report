\documentclass[a4paper]{cctart}
\usepackage{amsmath,bm,cases,cchead,fancyhdr}
\usepackage[affil-it]{authblk}
\usepackage{scrextend}
%\ffote[1.2em]{1.2em}{0.5em}{\thefootnotemark.\space}




\renewcommand{\vec}[1]{\bm{#1}}
\topmargin=0pt \oddsidemargin=0.66cm \evensidemargin=0.66cm
\textwidth=14cm \textheight=21cm
\renewcommand{\theequation}{\thesection\.arabic{equation}}
\newcommand{\eq}[1]{\begin{equation}{#1}\end{equation}}
\numberwithin{equation}{section} \pagestyle{fancy}
\markboth{清华大学文化素质讲座课程报告}{课程论文}
\title{文化素质讲座课程论文}
\author{高宸}
\date{2015.12}
\affil{电子工程系~无22班~2012011027}


\begin{document}
\maketitle
\begin{abstract}
文化素质讲座是一门创造性的课程,经过七个学期的学习,八个内容丰富、精彩绝伦的讲座,让我收获颇丰
。那么,接下来,
就让我来回顾下这八次美妙的旅程。
\end{abstract}

\section{内容综述}\label{1}
\subsection{贫富差别与社会公平}
清华大学社会学院院长李强给我们带来了名为“贫富差别与社会公平”的讲座,深刻阐述了对于贫富差距、
社会公平公正的理解。在讲座中,李强老师指出,对于公平公正的理解,有两种观念,一种观点认为,社会
分层结构本身就是不公正的,要实现公平必须消除分层结构,而另一种观点认为,分层结构的差异是无法消
除的,公平公正旨在保证人们进入这些机构的渠道和机会是公平公正的。

对于贫富差距的问题,李强老师介绍了衡量贫富差距的一个重要标准--基尼系数,基尼系数越大,贫富差距
就越大,目前中国的基尼系数大于0.5,表明目前还存在着较为严重的贫富差距问题。我们要正视贫富差距问
题,要警惕极少人垄断财富的极端,也要警惕民粹主义。

李强老师认为,中华民族的未来是广大中国民族的未来,不能是被少数人垄断财富的未来。
\subsection{中国传统文化价值理念的现代意义}
中国文化研究所所长刘梦溪给我们带来了名为“中国传统文化价值理念的现代意义”的演讲。在讲座中,刘
老师首先为我们讲述了中国传统文化在不同历史段落中的不同特色,先秦有诸子百家,在百家争鸣的时代,
产生了大量活跃的思想家,他们的思想涉及到世界观、价值观;秦汉时期,中国的学术思想主要表现为制度
文化,其中汉代以“经学”作为主要的学术思想,独尊儒术;到了三国魏晋南北朝时期,『玄学』成为主要学
术思想;隋唐时期,佛学发展到鼎盛时期;宋代的理学思想可谓集大成者,吸收了前代的各种思想,将中国
思想推向了一个高峰;王阳明的『心学』是明代的代表思想之一;清代的学术思想主要是考据学,体现朴实
无华的治学风格。

刘老师指出,中华传统文化中的价值理念永远不会过时,我们要集成并弘扬优秀的价值理念,视之为中华民
族的宝贵遗产。

\subsection{金融危机与乡村建设}
中国人民大学温铁军教授给我们带来了名为“金融危机与乡村建设”的演讲。早就耳闻温老师的名字,他在
经济建设颇有建树,这次讲座也是围绕农村经济建设展开。

温教授首先讲到,为什么中国在全球金融危机的背景中一直屹立不倒,“风景这边独好”,和中国雄厚的实
质资产是分不开的,中国的实质资产仍然处于扩张状态,金融资产尚未泡沫化。而现代化的背后,是代价,
虽然表面上我们只看到了利益,但是潜在着不可见的不安全因素。

对于乡村建设,温教授认为,农村问题是中国经济发展的重头戏,社会主义新农村建设是当前国家的一项重
大战略,但是目前处于攻坚阶段,遇到了很多难以解决的困难与挑战,但是,决不能因此就忽视“三农”问
题,对于农村建设绝不能动摇。
\subsection{中国说唱艺术的魅力}
著名相声表演艺术家姜昆老师给我们带来了名为“中国说唱艺术的魅力”的讲座。作为中国曲艺界的领军人
物,姜昆老师对于曲艺的定位是劳动的产物,即曲艺产生于最广大的劳动人民,所谓的曲艺,即是中华民族
各种“说唱艺术”的总称,中国的说唱艺术,异彩纷呈,包括丰富多样的艺术形式,“说”包括相声、快板、
评话等,“唱”包括评弹、琴书、大鼓等等。

姜昆老师给我们讲了他在曲艺创作中的一些趣事,并强调曲艺创作要深入群众,因为这是群众的艺术。在讲
座现场,姜昆老师给我们表演了《八扇屏》的一段,展现了扎实的艺术功底,《八扇屏》是一部运用“贯口”
手法(对口相声中常见的表现形式,也叫“背口”)的传统相声经典作品。

姜昆老师同时也讲到与时俱进的重要性,时代是发展的,曲艺艺术也要向前看,广大的曲艺艺术工作者,应
该努力创作,继承发扬中华民族的优秀艺术。
\subsection{电影创作中的历史观与人性表达}
著名导演冯小宁带来了名为“电影创作中的历史观与人性表达”的讲座。冯小宁先生是中国第五代导演的杰
出代表,从事电影创作几十年,自然有极深的体会。

冯老师认为,电影中体现的价值观会很直接地传递给观众,作为创作者,对于电影的价值观传达,自然必须有
自己的考量。冯老师的作品,很多反映近代的民族历史,有著名的『战争三部曲』,这些作品,并不仅仅停留
在叙事上,更从独特的视角,为我们展现一个个可歌可泣的民族传说。其背后,体现了对于战争、和平的思考。
冯小宁老师的电影,往往从民族冲突角度出发表现人性冲突,揭示了人类灵魂的特质。冯老师认为,电影是死
的,但是创作者是活的。

冯老师认为,只有真正用心去创作电影,观众才会真正地被电影所感染。
\subsection{深度学习}
清华大学工程力学系郑泉水教授给我们带来了“精深学习”的精彩演讲。讲座的核心是,如何提高清华学子
的综合素质与研究创新能力,郑教授以丹尼尔·科伊尔的一万小时天才理论\footnote{该理论认为,要成为
某个领域的专家,需要10000小时的练习}引入主题,认为在信息爆炸的今天,精深学习的概念显得尤为重要
。所谓精深学习,即是不再简单地将大量时间投入到某项工作中,而是在一定条件下,不断挑战自我,不断
犯错以获得进步和提高。

郑教授以自己的学习经历、爱因斯坦和比尔盖茨的成才经历做横向比较,具体说明了如何进行精深学习,以
及精深学习有怎样的成效。郑教授认为,清华的学生不能将时间浪费在刷学分绩上,而应该心怀的远大的理
想,在热爱的领域,展开挑战性的学习。而要做到这一点,也需要学校建立更完善的课程体系与评价体系。
郑教授认为知识有三重境界,由浅入深分别是:信息、技能、态度。知识只有是“活”的才有意义,“死”
的知识反而会束缚思维。郑老师以钱学森力学班的实践为例进一步说明了这一点。

在讲座结束后的互动环节,几位同学提出了很有前瞻性的问题,比如一位同学问是否应建立并推广免修制度,
而据我所知,当笔者写下本文时,已有部分院系对于部分课程实施了免修制度,这位同学的问题如今已经成
为了现实。
\subsection{美好青春我做主——青春红丝带校园行}
首都高校青春红丝带社团工作领导小组办公室副主任许建农先生给我们带来了关于艾滋病与毒品的讲座,此
次讲座针对于目前愈演愈烈的大学生艾滋病问题,普及艾滋病预防知识,让我们正确认识艾滋病。同时,在
很多时候,艾滋病是作为毒品的“衍生品”传播,所以徐老师细致讲解了毒品的种类、危害及其与艾滋病的
关系。

许老师指出,近年来,我国的艾滋病感染者和病人数量连年攀升,另一件恐怖的事实是,高校学生的艾滋病
感染者和病人数量的增加速率远高于平均速率,而这其中大部分是通过性传播,尤其是男男性传播。艾滋病
之所以特殊,因为艾滋病带来的不仅仅是病痛,更有白眼与冷落。对待艾滋病感染者,我们要给予足够的尊
重,保护隐私显得尤为重要,需要卫生行政部门、学校行政部门的共同努力。作为一名大学生,首先要充分
认识艾滋病的危害,洁身自好,遏制源头,同时也要提防一切误感染的可能性。

同时,徐老师为我们讲解了常见毒品的相关知识,毒品似乎离我们很远,其实很近,对待毒品,必须采取零
容忍的态度,因为传播、使用毒品,害人害己,天理不容。
\subsection{一带一路——历史视野与现实关照}
清华大学历史系教授张国刚教授为我们带来了关于“一带一路”的精彩讲座,此次讲座举办适逢我选修《党
的知识概论》,所以对于此次讲座有着特殊的感受。

所谓“一带一路”,即“丝绸之路经济带”与“21世纪海上丝绸之路”,由中国国家主席习近平在2013年出
访中亚和东南亚国家期间提出,受到国际社会的广泛高度关注。起始于古代中国的古代路上商业贸易路线“
丝绸之路”连接亚洲、非洲和欧洲,是东方与西方之间经济、政治、文化进行交流的重要道路;而“海上丝绸
之路”是古代中国与外国交通贸易和文化交往的海上通道,该路主要以南海为中心,所以又称“南海丝绸之路
”。

张老师的讲座内容对应于标题,从“历史视野”与“现实关照”两部分分别剖析“一带一路”。从历史上讲,
“一带一路”的提出不是信口开河,也不是纸上谈兵,它是一个可靠而且具有重大意义的战略,承接了古代中
国,具有深厚的人文关怀。同时,今非昔比,“一带一路”战略也与当今的时代背景相契合,从国内的角度,
目前改革步入深水区,到了矛盾集聚、风险挤压的关键期;从国外的角度,世界经济全球化、区域经济一体化
加快推进,世界经济同样到了转型升级的关键阶段。

\DeclareRobustCommand\CTeX{$\mathbb{C}$\kern-.05em\TeX{}}

\section{心得体会}\label{2}
对于X和Y两部分 ,我的感触较深,下面讲一讲我的心得体会。
\subsection{X}
目前存在着一个值得深思的矛盾,一方面大学生艾滋病形势严峻,另一方面,高校学生对于艾滋病的认识不够,
这个矛盾如果不能解决,艾滋病形势将变得越来越不可控。在讲座过后,我对周围同学对于艾滋病的了解程度
做了一次调查,调查结果显示,有很大一部分同学对于艾滋病的相关知识匮乏,而这部分同学,基本上没有接触
过艾滋病讲座或其他的科普途径。这一结果,证明了大力开展艾滋病知识科普工作的重要性。

这次讲座让我联想到我之前选修的《卫生与保健》这门课,这门课上给我们仔细讲解了艾滋病知识,但是在课堂
上,很多同学甚至只是签个到就走,这样的做法又怎么能学到知识?归根到底,还是因为大家觉得艾滋病离自己很远,
没有危机感。防范工作重在培养危机感,这点不容忽视。

从另一方面,我们要对艾滋病感染者和患者给予平等的态度,由于长期以来人们对艾滋病的偏见,认为艾滋病等价于
性病,等价于不洁的生活作风,但是事实上很多艾滋病的感染者事出有因,我们需要充分地理解他们,慢慢地扭转他
们广受歧视的现状。即使是他们之前犯了错而染上了艾滋病,我们也要以善意与热情对待他们,而不是冷漠与歧视。

同时,老师说道,男男传播是大学生艾滋病传播的一条主要途径,这就涉及到一些深层次的伦理问题,矛盾冲突会比较
激烈,但是我们要敢于直面矛盾,因为生命高于一切。

总的来说,艾滋病的防范需要多方的努力,从教育普及,到社会宣传,再到自身防范,每一环都极为重要。






\subsection{一带一路--历史视野与现实关照}
张老师在讲座上强调,一带一路从战略角度影响着世界秩序,每一个时代青年都应该对其有所了解。从政治角
度讲,只有周边政治稳定时,才能保证贸易的安全进行,从更深的层面理解,一带一路体现了中国的政治理念。
在讲座结束后,我对『一带一路』进行了调研,对这项战略有了更全面的了解和认识。
1)消化产能过剩。
通过合作投资推动周边国家基础设施建设,支持装备制造业出去,进而推进国内产能过剩行业到资源富集、市
场需求大的国家建立生产基地。以钢铁过剩大省河北省为例,“一带一路”为河北省的钢铁产业发展带来了新的
契机。众所周知,河北省的钢铁产量一直稳居全国第一,而“一带一路”带来的基础设施建设对于钢铁需求的巨
大拉动能力将有效地解决钢铁产能过剩的问题。对此,河北省出台的《关于主动融入国家“一带一路”战略促进
我省开放发展的意见》,明确鼓励产能过剩的相关企业,到境外建设产生基地,带动各方面的输出。不仅限钢
铁产业的输出,其他基础建设产业也受益于“一带一路”。例如目前大部分东盟国家和中亚地区工业化程度均不
高,基础设施较为落后,对铁路、电信、铁路等基础设备的需求量十分巨大。
2)获取资源多元化。
“一带一路”战略的实施可以帮助降低因过度依赖海上运输而带来潜在供给能力的脆弱性,并且与中亚国家能源
出口多元化战略在国家利益、技术力量和经济结构方面形成互补,重视与中国开展国际能源合作已成为“丝绸之
路”沿线国家合作的重要内容。我国新疆与中亚各国在能源与贸易上的互联互通方面进行了大量卓有成效的努力
,除相继建成中哈石油管道、中国-中亚天然气管道以及第二亚欧大陆桥外,各方还共同成立上海合作组成,中
国与哈萨克斯坦、乌兹别克斯坦等还建立国家级的经贸合作委员会,在加强能源俄合作方面取得显著的效果。中
亚各国与中国的能源合作已由原来的顾虑重重,发展到如今能源设备生产、管道运输系统等中下游领域的全面合
作。
3)开拓战略纵深和促进区域平衡发展。
加大对西部的开发,解决工业和基础设施集中问题。“一带一路”战略的提出,不仅将西部地区推到改革开放的
前沿,也是的第二轮西部大开发具有新鲜且充实的内容。对于基础设施、能源产业的建设,将会有效地带动沿线
各省的建设,在一定程度上对于区域平衡发展有着极大的推动作用。因为“丝绸之路经济带”可以调动区域内能源、
文化、旅游、工农业等各方面的资源,建立均衡合理的布局,有效地改善长期以来不平衡的高污染低效率经济体系,
实现完美转型。不仅如此,“一带一路”中的海上丝绸之路,在中国和东南亚国家临海港口城市之间建立有机的连接,
对于我国沿海港口城市的发展有着至关重要的引导。 
4)推动国际区域经济一体化发展。
以国际区域经济一体化为主要表现的国际区域经济合作,自20世纪50年代以来在欧盟的示范作用下进行了多种形式
的实践,20世纪90年代以来更得到迅猛发展,各种形式的区域经济一体化组织纷纷建立。国际区域经济一体化通常
被理解为,据有一定地缘关系的两个或者两个以上的国家或者地区,通过签订某种条约或协定,拟定共同的行动准
则和协调一致的政策,形成相互之间协作与支持的经济制度和市场。在世界经济研究中,“国际区域经济一体化”通
常简称为“区域经济一体化”。从世界经济发展趋势上看,区域经济一体化组织不仅数量越来越多,而且规模也越来
越大,说明区域经济合作是当今世界经济发展的主流。区域经济合作对世界经济的发展产生深刻影响,促使越来越
多的国家为谋取更大利益而加入各类区域性经济集团之中。
当今国际区域经济合作的发展总体呈现如下两个特点:一是发达国家发展程度高,而发展中国家发展程度低;二是西
欧及北美洲发展程度高,而亚洲、东欧、南美洲及非洲发展程度低。以德国和法国为主导的欧盟经济圈是当今世界区
域经济一体化程度最高的经济区。以美国为主导的北美自由贸易区是综合实力最强的经济区。相比之下,大多发展中
国家面临着来自安达国家主导的区域经济集团的挑战,中国与丝绸之路经济带沿线国家的经济合作在国际上处于滞后
位置。因此,在国际区域经济合作蓬勃发展的国际背景下,启动并加快丝绸之路经济带建设,对区域经济一体化在全
球领域发展有重要意义。
5)营造多极化趋势下国际经济发展的和平环境。
“二战”结束后,世界经济以美国为中心,单极化特征明显。随着20世纪60年代苏联经济实力大幅增强并成为世界第
二大经济体,世界经济向两极格局演变;70年代后,欧洲共同体继苏联之后恒伟世界经济中的又一极;80年代,日
本迅速崛起,成为第二经济大国,欧洲共同体进一步壮大,从而引起世界经济格局的又一重大转折;90年代初,随
着东欧剧变及苏联解体,苏联经济地位大幅降低,世界经济形成美国、日本、西欧三足鼎立的多极格局。进入21世
纪后,中国经济实力大幅增加,俄罗斯及许多发展中国家经济保持较高增速并成为新兴市场经济体。世界经济发展
的多极化趋势不断增强。世界经济有一个国家单方面“定夺”世界重要事务,转变为世界各国在多极格局中日益相互
依赖,相互竞争与相互协调。在错综复杂的多极化格局中,加强国际经济协调成为缓和国际经济冲突,扩大国际合
作的重要手段,并将对未来各国经济及世界经济的发展产生重大影响。
“一带一路”战略所涉及地区为亚欧经济的重要区域,沿线国家的经济协调和合作,对世界经济及自身经济的发展都
变得日益重要。各经济体发挥各自经济的优势,通过彼此相互开放,形成公平,统一的市场竞争环境,促进各种资源
的自由流动,调动各类经济主体发展积极性,形成互利共赢的发展模式,形成互联互通,以点带面,从线到片,逐步
形成大区域大合作的发展格局。
显然,“一带一路”战略实施将对全球经济版图长生重大影响,并有利于构建和谐世界发展的世界经济新格局。丝绸之
路经济带的战略构想尊重区域内各国人民自主选择发展道路的权利,主张通过加强政治沟通和战略互信,营造超越传
统国家关系模式、文明属性、制度差异、发展差距的新型国家关系,使参与丝绸之路经济带建设的国家成为共同发展、
共同安全的“好邻居、好伙伴、好朋友”。据此,“一带一路”战略实施有利于搭建灰机世界各国的和谐世界建设平台,
营造国际经济发展的和平环境。
总的来说,从国内层面,“一带一路”战略带来的内需可以有效地解决部分产业的产能过剩问题,并给我国的资源需求
提供更多更完善的路径,对于“一带”的沿带地区和“一路”的港口城市,将会带来天时地利;从国际层面,“一带一路”
战略是经济全球化的必然选择,同时是我国在国际上的经济地位的有效保障。
不论是近期的经济发展,还是长远的战略价值,“一带一路”都向我们证明了它的价值,“一带一路”继承自古代的“丝绸
之路”,其背后,还体现着厚重的文化底蕴和人文关怀,是历史的选择,蕴含着我国人民互惠合作的优良传统。我们有
理由相信,以“一带一路”为代表,中国将肩负越来越重要的国际责任,为构建国际合作的利益、命运和责任共同体而
继续努力着。

\subsection{$Noll$三原则}

(1){\kaishu 物质客观性原理}
(2){\kaishu 确定性(遗传性)原理}
(3){\kaishu 局部作用原理}


\section{课程意见和建议}\label{3}
转眼间,从刚开始参加文化素质讲座的学习到现在,已是好几年了,我忘不了第一次听讲座时的好奇与忐忑,
忘不了排队等主讲老师的兴奋,更忘不了这三年来,一次次文化素质讲座上老师谈经论道、思想与知识碰撞的火花。
在我看来,这门课的设立,是我辈之幸,是清华之幸!清华工科气太重了,这也是我们要想成长为世界一流大学必须
克服的。教书育人,不仅仅要教专业之术,更要培养一个完整的人,这样一个人,必然是需要懂『文化』的。
通识教育永远不能懈怠,对于清华而言,通识教育的重要性应该是和以『为祖国工作五十年』为口号的体育教育一个
层次的。
然而,在执行上,这门课程还存在着一些亟待改进之处。
1.缺乏有效的监督制度。很多同学往往是讲座开始前去刷卡,然后立刻离开,这种做法必须严格惩罚。
2.讲座预告中对于讲座内容缺乏足够的介绍,很多时候的预告只有讲座名称和主讲人的信息,没有对于讲座内容较为详细
的介绍,同学们看到预告比较迷茫。
3.需要及时的学习反馈。学生在听完讲座后,需要设置作业或者是讨论对讲座内容进行思考与学习,及时地吸收讲座内容。
多年之后,在清华学习的很多课程我都会忘记,但是我一定将一门课铭记在心,它的名字叫做『文化素质讲座』!


\subsection{简单物质的本构方程}


\section{结论}

\section{致谢}
在此本人表示感谢!

\begin{thebibliography}{9}
\bibitem{Gurtin}Gurtin~M~E, 1981, An Introduction to Continuum
Mechanics, Academic Press.
\bibitem{kzb1}Kuang~Z~B, 1990, Integral constitutive equation
of elastic-plastic materials. Acta Mechanica Solida
Sinica,3:245-262.
\bibitem{Miller}Miller~A~K, 1987, Unified constitutive equations
for creep and plasticity. Elsevier Applied Science World
Publishing Coporation, New York.
\bibitem{Tanaka}Tanaka~T~G, Miller~A~K, 1988, Development of a
method for integrating time-dependent constitutive equations with
large, small or negative strain rate sensitivity. Int J Numerical
Methods in Engng, 26:2457-85.
\bibitem{kzb2} 匡震邦,1989,非线性连续介质力学基础,西安交通大学出版社.
\end{thebibliography}
\end{document}
